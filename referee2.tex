\reviewer

\comment{The abstract should be re-written to better describe some details of study objective, methods and main findings.}

\response{We appreciate the reviewer's suggestion and have added a more thorough description to the abstract.}

\comment{Literature review is not up to date, especially related to fall risk determination. 
The statement `Fall risk is currently determined by medical history, tests of flexibility and strength, or simply a physician's intuition' does not truly reflect the current knowledge and practice. 
The citation, Perell et al, 2001, is out of date.}

\response{We thank the reviewer for pointing out the gaps in our literature review. We have added both clinical guidelines developed in $2013$ used by the UK National Institute for Health and Care Excellence and a review of clinical tools to assess fall risk in hospital settings to demonstrate that the inadequacy of current predictive tools has resulted in clinical judgement and medical history being the widespread determinant of fall risk.}

\comment{There is no mention of any clinical assessments of fall risk.}

\response{We thank the reviewer for their assessment, and have included the guidelines set forth by the UK National Institute for Health and Care Excellence and a review of clinical tools to assess fall risk in hospital settings.}

\comment{Although I agree that the ability to perform sit-to-stand is key to maintaining independence and quality of life, the transition to this specific daily activity in the Introduction requires to be re-worked.} 

\response{We thank the reviewer for pointing out this gap in narrative flow. We have elaborated on the transition from the previous paragraph in response to this comment and comment 1.14.}

\comment{There is a lack of justification on the `infeasibility' of empirically determining stability.}

\response{We appreciate the reviewer's perspective, and have expanded on the practical infeasibility of thoroughly understanding individual stability via empirical perturbation.}

\comment{Please consider changing the organization and writing style of the Introduction, especially the past paragraph. Citing the results figures in the Introduction is inappropriate.}

\response{We thank the reviewer for their suggestion, and have removed the citation of figures in the Introduction.}
% While we are unsure of the reviewer's preferred organization and writing style, we have significantly revised this section.}

\comment{Theoretical description of the dynamical model is difficult to read and understand. Are those assumptions made in controller designs supported by any previous studies.}

\response{We appreciate the reviewer's concern, and have added additional language related this comment and to comments 1.5, 1.6, 1.7, 1.8, 1.13 to make the description of the dynamical model more clear.}

\comment{The connection between the IPM or DIP during standing and STS activity investigated in this study was not clear. Which model was used to compute the SB?}

\response{We appreciate the reviewer's confusion. 
These models differ in complexity, and the same SB calculation was done for each model to determine whether increased complexity conferred additional accuracy and reliability. 
We have substantially revised the Methods section in response to this comment and comments 1.5, 1.9, 1.14, and 1.20.}

\comment{In the Results, the section described ROSa was excessive and should be better described in the Methods in advance.}

\response{We agree with the reviewer that the description of ROSa was excessive. 
This was due to the somewhat complex details of directly comparing the SB to the ROSa method. 
We originally presented the ROSa description in the Results section so as not to give the impression that we developed the ROSa method ourselves. 
Since the details of this comparison are not integral to the SB framework, we have opted to shift this description to the figure caption.}

\comment{It is not clear how the OS and MT was distinguished using the SB?}

\response{While we appreciate the reviewer's concern, we are a bit confused as to what the reviewer is asking.
If the reviewer is asking ``Why is the QS strategy assumed to be more stable than the MT strategy:'' 
As observed in the literature, the QS strategy is more robust to perturbation than its MT counterpart, which means that the SB for QS should be larger than the one for MT. 
If the reviewer is asking ``How do you use SB to determine whether a subject us performing an MT or QS strategy:'' 
SB was not used to determine the STS strategy. 
Instead, as described in the Methods, during the experiment, we designated whether the subject should perform the QS or MT strategy before each STS trial.
Each strategy, as described in Results, was found to have distinct ranges of peak horizontal acceleration.
We have added additional language to further clarify both of these potential concerns.}
