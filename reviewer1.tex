\reviewer

\comment{This paper demonstrates combining multiple fiber reinforced elastomeric
actuators in parallel to achieve multi-dimensional soft actuation. The
authors present a mathematical model based on the notion of fluid
Jacobian to predict the spatial forces exerted by these actuators in
parallel. The force zonotope of the system of three fiber reinforced
elastomeric actuators is of particular interest. The force zonotope of
the system obtained from experiments is consistent with the
mathematical model. The paper encourages further applications of the
model presented in the paper to complex soft robotics manipulator.}
\response{We thank the reviewer for their comments.}

\comment{The introduction sufficiently explains the significance of the paper by
stating possible applications. However, it would help to also
demonstrate an application of the fiber reinforced actuator in the
paper.}
\response{We acknowledge the need to better motivate the potential applications of the fiber-reinforced soft actuator specifically. We have added some examples to the introduction to highlight the way it has been used in applications by other researchers.}

\comment{The paper is generally written and well organized, however, an
explanation of the experimental setup and the behavior of the system
using a schematic will be useful.}
\response{We agree that further clarification of the experimental setup is necessary. The caption to Fig. 8 has been updated to explain the system in more detail, and more components have been labeled in the photograph.}

\comment{The largest concern I have is that the experimental setup
(Figure 7, 8) constrains the rotation of the end effector in x and y
direction. If this constraint was absent, the actuator would bend in x
and y direction. The authors should justify the validity of
constraining the rotation of the end plate along x and y axis. Is this
practical for any applications of soft robotics?}
\response{We understand the reviewers concern that a soft robot constrained as in our experiment has no practical value, but believe that this constraint is appropriate within the scope of this paper. The purpose of this paper is to present a model that describes the forces generated by parallel combinations of soft actuators. In order to measure such forces and evaluate the model it is necessary to inhibit motion. Thus, in the experimental setup, we actually constrain the end effector in all directions, not just x and y. We chose, however, only to \emph{measure} the force and moment about the z-axis for several reasons. First, these are the directions most closely aligned with the force directions of the individual FREEs. Secondly, two force measurements can be coherently displayed on a 2-axis plot, making the data more presentable to the reader.}

\comment{A configuration of fiber reinforced actuators in series is of
interest for applications such as soft endoscopes. The authors should
address how their modeling approach could be used for fiber reinforced
elastomeric actuators in series.}
\response{We agree that fiber reinforced actuators in series are of significant interest for many applications, but that extending our modeling approach to series combinations falls outside the scope of this paper since it does not generalize as readily from the kinetic model of a single actuator. We have added a disclaimer explicitly define the intended scope of this paper.}

\comment{A Stewart platform consists of six prismatic linear actuators
in parallel to achieve six degrees of freedom. Figure 1 (bottom right)
shows an application of fiber reinforced elastomeric actuator in
Stewart platform. However, since single DOF linear actuators are
sufficient for a Stewart platform, it is unclear how the proposed multi
DOF actuators are relevant. The author should clearly state the
advantage of having extra DOFs in the actuators for Stewart platform.}
\response{As the reviewer points out, in a Stewart platform 1 DOF actuators are sufficient to generate 6 DOF forces, however, each actuator must also be paired with 5 joints to overcome the kinematic constraints imposed by the actuators themselves. Our thought was that a similar arrangement of FREE actuators could generate 6DOF forces without the need for extra joints, i.e. mechanical complexity. However, to avoid any confusion and save space we have decided to remove the Stewart platform example from the paper in lieu of a simpler example.}

\comment{The actuators demonstrated in the paper are already called
fiber-reinforced soft actuators (FRSA) in the literature [1][2]. The
author should justify the reason for using a new acronym, fiber
reinforced elastomeric actuators (FREE), which will only serve to
confuse the reader.}
\response{We thank the reviewer for bringing up this possible point of confusion. While we maintain the use of the FREE acronym as it has been used in literature that precedes the papers mentioned [13][14], we have added text to acknowledge the actuator's alternate name.}

\comment{A video would go a long way to help in communicating the
experimental results.}
\response{TBD}