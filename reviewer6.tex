\reviewer

\comment{This work presents a zonotope framework on using multiple soft
actuators to form an assembly to generation multi-DOF motions.
The paper was clearly written with easily comprehensible graphics and
equations.}
\response{We thank the reviewer for their kind comments.}

\comment{The concept design of the actuator itself has been previously published
(widely known by the soft robotic community). The concept of using
multiple (three, in particular) actuators to form platforms is also
previously published (as mentioned by the authors). The main
contribution of this work would be the combination of multiple
actuators of DIFFERENT fiber alignments to achieve new features. However, the poor repeatability of soft actuators, considering the
highly manual fabrication procedures, was well-known. Therefore,
calibration and validation of each actuator unit would be essential to
justify for the platform's performances.}
\response{}
% \response{We agree that currently one of the major shortcomings of these actuators is the performance errors introduced by fabrication defects and unmodeled material effects such as hysteresis. However, we contend that the experiment presented is adequate to justify the performance of the proposed modeling approach}

\comment{Single actuator was not thoroughly presented nor tested. Considering
the nearly 20\% error (1.5N versus 6N total range tested), strong
evidences are required to split the actuator variation from structural
variation of the platform, and further from other factors as mentioned
in the paper. The authors extensively cited [12], previous publication from the same
group, but no force or repeatibility analyses were available in that
earlier paper, either.}
\response{}

\comment{For the zonotope framework, how was gravity dealt with in the
experimental platform? Given the passive flexibility of soft actuators,
gravity, or self-weight would affect significantly the behavior of the
system, especially in horizontal alignments, but neither was mentioned
in this paper in its current form.}
\response{We thank the reviewer for their comment, and hope we can justify our omission of a discussion of gravity. In this work the effect of gravity is assumed to be negligible. While gravity acting along the body of the FREE does induces a downward force, the magnitude of said force is too small to induce any perceptible displacement along the length of the FREEs used in testing. Even when supported in a cantilever configuration, the stiffness of the rubber FREE wall prevents any perceptible deflection due to the force of gravity. Furthermore, the entire body of each FREE has a mass of less than 0.008 kg which amounts to a total gravitational force of less than 0.08 N. Therefore, even if some component of this force were to act along the direction of the forces being measured, the error induced by this effect would account for less than 3\% of the maximum error observed. We did not include a discussion of gravity in the paper in order to avoid inundating the reader with details we deemed irrelevant.}

\comment{Application. With the poor repeatibility of each actuator unit, the
overall platform realizing the proposed zonotope would most likely be
prone to large errors. Although mentioning that this work may lay the
foundation to further applications, the authors would need to identify
2-3 potential applications where such systems or concepts could be
utilized, in order to provide a focused view of the further
developments of this work.}
\response{We agree that more examples of potential uses of the force zonotope would better motivate the utility of the work. We have added several such examples to the conclusion section.}

\comment{Modeling section was extensive, which could potentially be reduced
for a more detailed results/experiment section.}
\response{}